 \documentclass[a4paper, 12pt]{article}
 \usepackage{amsmath}
\usepackage{graphicx}
% \usepackage{subcaption}
% \usepackage{longtable}
\newcommand{\thema}{Die drei grundlegenden Sprachen einer Webseite}
\title{
	Beruf- und Arbeitspädagoge\\
	 \thema
	\newline
	\begin{figure}[h!]
		\vspace{15mm}
	\includegraphics[scale=0.2, width=\linewidth]{../pics/htmlCssJsWordwordle.png}
\end{figure}
	\vfill
}
\author{
	Josef Sebastian Duschl
}
\date{\today}


\begin{document}


%\section{Section}
%Dummy Text

%\subsection{Subsection}
%Dummy Text

    \pagenumbering{gobble}
    \maketitle
    \newpage
    \pagenumbering{arabic}

    \tableofcontents % Inhaltsangabe
    \newpage

	\section{Einleitung}
	Konzept zur Darstellung einer Ausbildungssituation im Rahmen der Ausbildereignungsprüfung

    \subsection{Methodik}
        Didaktisches Konzept für die praktische Prüfung in Form einer Präsentation
    
    \subsection{Thema}
    \thema
    
    \subsection{Name, Matrikelnummer}
    Josef Sebastian Duschl \hfill Matrikelnummer: 863219
    
    \subsection{Eigenständigkeitserklärung }
    Ich versichere, dass dieses Konzept von mir selbständig erstellt und noch bei keiner anderen Prüfung vorgelegt wurde.\\
    
    % Ort Datum Unterschrift
    \vspace{1,5 cm} 
    \begin{tabular}{p{7cm}p{.5cm}l}
    	\dotfill \\ 
    	Ort, Datum
    \end{tabular}% 
    \hfill 
    \begin{tabular}{p{7cm}p{.5cm}l}
    	\dotfill \\ 
    	Unterschrift
    \end{tabular}

   \subsection{Ausbildungsberuf}
	Fachinformatiker/innen(m/w/divers) für Anwendungsbetreuung
	
	\subsection{Einordnung in den Ausbildungsrahmenplan}
	§5 Abs. 1 Nr. 2 der Ausbildungsverordnung über die Berufsausbildung Fachinformatiker/innen(m/w/divers) für technische Systeme
	\newline
	Ausbildungsjahr: 1\\
	Ausbildungsmonat: 1\\
     
    \subsection{Ziel der Ausbildungssituation (operationalisiert)}     
        Nach der Präsentation kennt der Auszubildende die Basics der drei grundlegenden Sprachen einer Webseite. Den Aufbau, die Positionierung und die Validierung.
   
    \subsection{Zu vermittelnder Inhalt berufsspezifischer Fachqualifikation}
    	Die Vermittlung des Ausbildungsinhalts findet in Anlehnung an die Verordnung der Berufsausbildung zum Fachinformatiker/innen(m/w/divers) vom 5. März 2020 statt. Der Ausbildungsrahmenplan gibt berufsprofilgebende Fertigkeiten, Kenntnisse und Fähigkeiten vor. Laut §4 Absatz 3 Nummer 1 müssen das Konzipieren und Umsetzen von Kundenspezifischen Softwareanwendungen im ersten Lehrjahr vermittelt werden. ( siehe hierzu Anlage Ausbildungsrahmenplan). Hinzu kommt, dass Vorgehensmodelle und -methoden sowie Entwicklungsumgebungen und -bibliotheken auszuwählen und einzusetzen sind. Das umfasst die Analyse- und Designverfahren. Dazu zählen unter anderem die ergonomische Gestaltung und anzupassende Kundenanforderungen. (Siehe hierzu Anlage Ausbildungsrahmenplan). Es ist wichtig zu wissen, worauf Bibliotheken basieren und wie Sie eingebunden werden können. Elementar ist zu wissen, auf welcher Programmiersprache eine ausgewählte Bibliothek basiert. 
   	
   	\subsection{Lösungsalternativen}
   	Die Präsentation bietet sich zum Erlernen der Grundstruktur, der visuellen Gestaltung und weitere Änderungen nach dem Laden einer Webseite. Sie erleichtert durch den selbsterklärenden Programmcode den Einstieg in die Webtechnologie. HTML, CSS und Javascript finden immer mehr Bedeutung in gängige Anwendungen wie Exel oder Google Apps. Es lassen sich einfache Scripte selbst schreiben, die dann beispielsweise visuell als Präsentation in Google Slide erscheinen können. Eine Unterweisung mit der Lehrgesprächsmethode ungeeignet, weil es hier um einen groben Überblick der Technologien geht. Die 4-Stufen Methode wäre denkbar, aber selbst hier gibt es verschiedene Ansätze und Lösungen. Hier müsste anhand von Vorteilen und Nachteilen abgewogen werden. Dies wird schnell unübersichtlich.

	\subsection{Begründung der eigenen Lösung}
	Die Präsentation wurde gewählt, weil es um eine reine Übermittlung des Lernstoffes geht. Es soll vermittelt werden, wofür, wozu welche Technologie verwendet werden kann. Selbsterklärende Codebeispiele unterstützen das Verständnis des Auszubildenden. Dadurch wird der Bezug zu den drei Grundlegenden Sprachen hergestellt. Zudem kann das Erlente visuell und auditivg wahrgenommen werden.
	
   
   \subsection{Lernort}
   	Schulungsraum
   
   \subsection{Hilfsmittel}
   	Beamer, Laptopt, Präsentationsfolien, Handout, Tafel, Flipchart, Leere Seiten
   	
   	\subsection{Zeitrahmen}
     der Zeitrahmen der Präsentation beträgt ca. 15 Minuten.
   
	\subsection{Allgemeine Hinweise}
	Bitte schalten Sie ihre Mobilgeräte Stumm.

	\section{Ablaufstruktur / Gliederung der Präsentation}
	Nachfolgend ist die Gliederung der Präsentation aufgelistet.
	
	\subsection{Schulungsraum vorbereiten}
		\begin{itemize}
			\item Die Präsentation wird vorbereitet
			\item Die Materialien werden ausgelegt
		\end{itemize}
	\subsection{Eröffung}
		\begin{itemize}
			\item Begrüßung\\
			Der Auszubildende wird im Schulungsraum freundliche Begrüßt.
			\item Vorstellung\\
			Ich stelle mich selber vor und nenne mein Aufgabengebiet im Unternehmen.
			\item Schaffen einer guten Atmospähre und Befangenheit abbauen:\\
			Da der Azubi erst seit einer Woche im Unternehmen ist, kann man die Frage äußern wie es Ihm derzeit ergangen ist und er sich im Unternehmen zurechtfindet.
			\item Nennen des Lernziels\\
			Dem Auzubildenden wird das Lernziel genannt. Um sich im Programmcode zurecht zu finden ist der Aufbau der einzelnen Sprachen elementar. Ist dies klar, kann ein Programmcode einer Webseite leichter gelesen werden und das Verständnis dazu geschult.
			\item Wecken von Interesse\\
			Die Sprachen lassen sich auch in einen anderen Kontext verwenden. Zum Beispiel als App oder als Präsentation. Sie bilden die Basics worauf vieles im Web aufbaut.
		\end{itemize}
	\section{Hauptteil}
	\begin{itemize}
		\item Inhaltsvermittlung\\
			Der Inhalt wird anhand der Präsentation vermittelt. Zu Beginn der Präsentation wird eine allgemeine Frage zum Thema "Webseite" gestellt. Der Auzubildende kann sich nun in das Thema eindenken. Hinzu kommt, dass der Wissenstand festgestellt werden kann. Darauf folgend wir die Präsentation wie eine Schulung vorgetragen.
	\end{itemize}

\newpage
\begin{table}[h!]
	\begin{center}
		\caption{Inhaltsvermittlung.}
		\label{tab:table1}
		\begin{tabular}{|l|p{10cm}|}
			\textbf{Fremdwort} & \textbf{Erklärung}\\
			\hline
			operationalisiert &[1] theoretische Begriffe und Hypothesen im Sinn ihrer empirischen Überprüfbarkeit umformulieren, so dass im Einzelnen überprüfbare Zielvorgaben und Schritte vorliegen\\
		\end{tabular}
	\end{center}
\end{table}

        
\newpage        
    \section{Glossar}
    %\textwidth
   \begin{table}[h!]
   \begin{center}
   	\caption{Multirow table.}
   	\label{tab:table1}
   	\begin{tabular}{|l|p{10cm}|}
   		\textbf{Fremdwort} & \textbf{Erklärung}\\
   		\hline
   		operationalisiert &[1] theoretische Begriffe und Hypothesen im Sinn ihrer empirischen Überprüfbarkeit umformulieren, so dass im Einzelnen überprüfbare Zielvorgaben und Schritte vorliegen\\
   	\end{tabular}
   \end{center}
\end{table}



\newpage

    \paragraph{Paragraph}
        Some more text

    \subparagraph{Subparagraph}
        Even more text

    \section{Another seciton}

\begin{equation*}
    f(x) = x^2
\end{equation*}

This formula $f(x) = x^2$ is an example.

\begin{equation*}
    1 + 2 = 3
\end{equation*}

\begin{equation*}
    1 = 3 - 2
\end{equation*}

\begin{align*}
    1 + 2 &= 3 \\
    1 &= 3 - 2
\end{align*}

\begin{align*}
    f(x) & = x^2\\
    g(x) &= \frac{1}{x}\\
    F(x) &= \int^a_b \frac{1}{3}x^3
\end{align*}

\begin{equation*}
    \frac{1}{\sqrt{x}}
\end{equation*}

[
$\begin{matrix}  	%"$" hier Wichtig
    1 & 0\\
    0 & 1
\end{matrix}$  	% "$" hier Wichtig
]

$\left[				%"$" hier Wichtig
\begin{matrix}  
    1 & 0\\
    0 & 1
\end{matrix} 
\right]$  			% "$" hier Wichtig

$\left(\frac{1}{\sqrt{x}}\right)$

\end{document}

      