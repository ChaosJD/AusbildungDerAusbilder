 \documentclass{article}
 \usepackage{amsmath}
 \usepackage{graphicx}
 \usepackage{subcaption}
 \usepackage{longtable}
\newcommand{\thema}{Die drei grundlegenden Sprachen einer Webseite}
\title{
	Beruf- und Arbeitspädagoge\\
	 \thema
	\newline
	\begin{figure}[h!]
		\vspace{15mm}
	\includegraphics[scale=0.2, width=\linewidth]{../pics/htmlCssJsWordwordle.png}
\end{figure}
	\vfill
}
\author{
	Josef Sebastian Duschl
}
\date{2020-06-08}


\begin{document}


%\section{Section}
%Dummy Text

%\subsection{Subsection}
%Dummy Text

    \pagenumbering{gobble}
    \maketitle
    \newpage
    \pagenumbering{arabic}

    \tableofcontents % Inhaltsangabe
    \newpage

	\section{Einleitung}
	Konzept zur Darstellung einer Ausbildungssituationim Rahmen der Ausbildereignungsprüfung
	\subsection{Thema}
	\thema

    \subsection{Methodik}
        Gewählte Methodik: Präsentation
     
    \subsection{Ziel der Ausbildungssituation(operationalisiert)}     
        Nach der Präsentation kennt der Auszubildende die Basics der drei grundlegenden Sprachen einer Webseite. Den Aufbau, die Positionierung und Validierung.
        
    \subsection{Ausbildungsberuf}
    Fachinformatiker und zur Fachinformatiker
    
    \subsection{Einordnung in den Ausbildungsrahmenplan}
    §5 Abs. 1 Nr. 1 der Verordnung über die Berufsausbildung  Fachinformatiker und zur Fachinformatikerin für technische Systeme
    \newline
    Ausbildungsjahr: 2\\
    Ausbildungsmonat: 2\\
   
   \subsection{Lernort}
   Schulungsraum
   
   \subsection{Hilfsmittel}
   Handout, Tafel oder Flipchart, imagepress.js, Muster des Berichtsheftes??
   
	\subsection{Allgemeine Hinweise}
	Bitte schalten Sie ihre Mobilgeräte Stumm oder automatisieren Sie es.
	
	\subsection{Geplanter Ablauf}
	\begin{enumerate}
		\item Was sind die Big 3. %Wie sieht sowas aus? Beispielcode
		\item Wie sieht die aktuelle Situation aus
		\item Ausschnitt aus der Präsentation % imagepress.js or remark.js
		\item Ausblick
	\end{enumerate}
\newpage
\section{Ablauf}
\begin{table}[h!]
	\begin{center}
		\caption{Multirow table.}
		\label{tab:table1}
		\begin{tabular}{|l|p{10cm}|}
			\textbf{Fremdwort} & \textbf{Erklärung}\\
			\hline
			operationalisiert &[1] theoretische Begriffe und Hypothesen im Sinn ihrer empirischen Überprüfbarkeit umformulieren, so dass im Einzelnen überprüfbare Zielvorgaben und Schritte vorliegen\\
		\end{tabular}
	\end{center}
\end{table}

        
\newpage        
    \section{Glossar}
    %\textwidth
   \begin{table}[h!]
   \begin{center}
   	\caption{Multirow table.}
   	\label{tab:table1}
   	\begin{tabular}{|l|p{10cm}|}
   		\textbf{Fremdwort} & \textbf{Erklärung}\\
   		\hline
   		operationalisiert &[1] theoretische Begriffe und Hypothesen im Sinn ihrer empirischen Überprüfbarkeit umformulieren, so dass im Einzelnen überprüfbare Zielvorgaben und Schritte vorliegen\\
   	\end{tabular}
   \end{center}
\end{table}


\newpage

    \paragraph{Paragraph}
        Some more text

    \subparagraph{Subparagraph}
        Even more text

    \section{Another seciton}

\begin{equation*}
    f(x) = x^2
\end{equation*}

This formula $f(x) = x^2$ is an example.

\begin{equation*}
    1 + 2 = 3
\end{equation*}

\begin{equation*}
    1 = 3 - 2
\end{equation*}

\begin{align*}
    1 + 2 &= 3 \\
    1 &= 3 - 2
\end{align*}

\begin{align*}
    f(x) & = x^2\\
    g(x) &= \frac{1}{x}\\
    F(x) &= \int^a_b \frac{1}{3}x^3
\end{align*}

\begin{equation*}
    \frac{1}{\sqrt{x}}
\end{equation*}

[
$\begin{matrix}  	%"$" hier Wichtig
    1 & 0\\
    0 & 1
\end{matrix}$  	% "$" hier Wichtig
]

$\left[				%"$" hier Wichtig
\begin{matrix}  
    1 & 0\\
    0 & 1
\end{matrix} 
\right]$  			% "$" hier Wichtig

$\left(\frac{1}{\sqrt{x}}\right)$

\end{document}

      