 \documentclass[a4paper, 12pt]{article}
 \usepackage{amsmath}
\usepackage{graphicx}
%\usepackage{pdfpages}
% \usepackage{subcaption}
% \usepackage{longtable}
\newcommand{\thema}{Die drei grundlegenden Sprachen einer Webseite}
\title{
	Beruf- und Arbeitspädagoge\\
	 \thema
	\newline
	\begin{figure}[h!]
		\vspace{15mm}
	\includegraphics[scale=0.2, width=\linewidth]{../pics/htmlCssJsWordwordle.png}
\end{figure}
	\vfill
}
\author{
	Josef Sebastian Duschl
}
\date{\today}


\begin{document}


%\section{Section}
%Dummy Text

%\subsection{Subsection}
%Dummy Text

    \pagenumbering{gobble}
    \maketitle
    \newpage
    \pagenumbering{arabic}

    \tableofcontents % Inhaltsangabe
    \newpage

	\section{Einleitung}
	Konzept zur Darstellung einer Ausbildungssituation im Rahmen der Ausbildereignungsprüfung

    \subsection{Methodik}
        Didaktisches Konzept für die praktische Prüfung in Form einer Präsentation
    
    \subsection{Thema}
    \thema
    
    \subsection{Name, Matrikelnummer}
    Josef Sebastian Duschl \hfill Matrikelnummer: 863219
    
    \subsection{Eigenständigkeitserklärung }
    Ich versichere, dass dieses Konzept von mir selbständig erstellt und noch bei keiner anderen Prüfung vorgelegt wurde.\\
    
    % Ort Datum Unterschrift
    \vspace{1,5 cm} 
    \begin{tabular}{p{7cm}p{.5cm}l}
    	\dotfill \\ 
    	Ort, Datum
    \end{tabular}% 
    \hfill 
    \begin{tabular}{p{7cm}p{.5cm}l}
    	\dotfill \\ 
    	Unterschrift
    \end{tabular}

   \subsection{Ausbildungsberuf}
	Fachinformatiker/innen(m/w/divers) für Anwendungsbetreuung
	
	\subsection{Einordnung in den Ausbildungsrahmenplan}
	§5 Abs. 1 Nr. 2 der Ausbildungsverordnung über die Berufsausbildung Fachinformatiker/innen(m/w/divers) für technische Systeme
	\newline
	Ausbildungsjahr: 1\\
	Ausbildungsmonat: 1\\
     
    \subsection{Ziel der Ausbildungssituation (operationalisiert)}     
        Nach der Präsentation kennt der Auszubildende die Basics der drei grundlegenden Sprachen einer Webseite. Den Aufbau, die Positionierung und die Validierung.
   
    \subsection{Zu vermittelnder Inhalt berufsspezifischer Fachqualifikation}
    	Die Vermittlung des Ausbildungsinhalts findet in Anlehnung an die Verordnung der Berufsausbildung zum Fachinformatiker/innen(m/w/divers) vom 5. März 2020 statt. Der Ausbildungsrahmenplan gibt berufsprofilgebende Fertigkeiten, Kenntnisse und Fähigkeiten vor. Laut §4 Absatz 3 Nummer 1 müssen das Konzipieren und Umsetzen von Kundenspezifischen Softwareanwendungen im ersten Lehrjahr vermittelt werden. ( siehe hierzu Anlage Ausbildungsrahmenplan). Hinzu kommt, dass Vorgehensmodelle und -methoden sowie Entwicklungsumgebungen und -bibliotheken auszuwählen und einzusetzen sind. Das umfasst die Analyse- und Designverfahren. Dazu zählen unter anderem die ergonomische Gestaltung und anzupassende Kundenanforderungen. (Siehe hierzu Anlage Ausbildungsrahmenplan). Es ist wichtig zu wissen, worauf Bibliotheken basieren und wie Sie eingebunden werden können. Elementar ist zu wissen, auf welcher Programmiersprache eine ausgewählte Bibliothek basiert. 
   	
   	\subsection{Lösungsalternativen}
   	Die Präsentation bietet sich zum Erlernen der Grundstruktur, der visuellen Gestaltung und weitere Änderungen nach dem Laden einer Webseite. Sie erleichtert durch den selbsterklärenden Programmcode den Einstieg in die Webtechnologie. HTML, CSS und Javascript finden immer mehr Bedeutung in gängige Anwendungen wie Exel oder Google Apps. Es lassen sich einfache Scripte selbst schreiben, die dann beispielsweise visuell als Präsentation in Google Slide erscheinen können. Eine Unterweisung mit der Lehrgesprächsmethode ungeeignet, weil es hier um einen groben Überblick der Technologien geht. Die 4-Stufen Methode wäre denkbar, aber selbst hier gibt es verschiedene Ansätze und Lösungen. Hier müsste anhand von Vorteilen und Nachteilen abgewogen werden. Dies wird schnell unübersichtlich.

	\subsection{Begründung der eigenen Lösung}
	Die Präsentation wurde gewählt, weil es um eine reine Übermittlung des Lernstoffes geht. Es soll vermittelt werden, wofür, wozu welche Technologie verwendet werden kann. Selbsterklärende Codebeispiele unterstützen das Verständnis des Auszubildenden. Dadurch wird der Bezug zu den drei Grundlegenden Sprachen hergestellt. Zudem kann das Erlernte visuell und auditiv wahrgenommen werden.
	
   
   \subsection{Lernort}
   	Schulungsraum
   
   \subsection{Hilfsmittel}
   	Beamer, Laptop, Präsentationsfolien, Handout, Tafel, Flipchart, Leere Seiten
   	
   	\subsection{Zeitrahmen}
     der Zeitrahmen der Präsentation beträgt ca. 15 Minuten.
   
	\subsection{Allgemeine Hinweise}
	Bitte schalten Sie ihre Mobilgeräte Stumm.

	\section{Ablaufstruktur / Gliederung der Präsentation}
	Nachfolgend ist die Gliederung der Präsentation aufgelistet.
	
	\subsection{Schulungsraum vorbereiten}
		\begin{itemize}
			\item Die Präsentation wird vorbereitet
			\item Die Materialien werden ausgelegt
		\end{itemize}
	\subsection{Eröffung}
		\begin{itemize}
			\item Begrüßung\\
			Der Auszubildende wird im Schulungsraum freundliche Begrüßt.
			\item Vorstellung\\
			Ich stelle mich selber vor und nenne mein Aufgabengebiet im Unternehmen.
			\item Schaffen einer guten Atmosphäre und Befangenheit abbauen:\\
			Da der Azubi erst seit einer Woche im Unternehmen ist, kann man die Frage äußern wie es Ihm derzeit ergangen ist und er sich im Unternehmen zurechtfindet.
			\item Nennen des Lernziels\\
			Dem Auszubildenden wird das Lernziel genannt. Um sich im Programmcode zurecht zu finden ist der Aufbau der einzelnen Sprachen elementar. Ist dies klar, kann ein Programmcode einer Webseite leichter gelesen werden und das Verständnis dazu geschult.
			\item Wecken von Interesse\\
			Die Sprachen lassen sich auch in einen anderen Kontext verwenden. Zum Beispiel als App oder als Präsentation. Sie bilden die Basics worauf vieles im Web aufbaut.
		\end{itemize}
	\section{Hauptteil}
	\begin{itemize}
		\item Inhaltsvermittlung\\
			Der Inhalt wird anhand der Präsentation vermittelt. Zu Beginn der Präsentation wird eine allgemeine Frage zum Thema "Webseite" gestellt. Der Auszubildende kann sich nun in das Thema eindenken. Hinzu kommt, dass der Wissensstand festgestellt werden kann. Darauf folgend wir die Präsentation wie eine Schulung vorgetragen.
	\end{itemize}

\newpage
\begin{table}[h!]
	\begin{center}
		\caption{Inhaltsvermittlung 1}
		\label{tab:table1}
		\begin{tabular}{|p{8cm}|p{8cm}|}
			\textbf{Präsentationsthemen} & \textbf{Zweck}\\
			\hline
			1. Allgemeine Frage zum Thema Webseite  & Der Auszubildende soll sich Gedanken zum Thema einer Webseite machen. Der Wissensstand wird festgestellt\\
			\hline
			2. Die drei grundlegenden Sprachen einer Webseite & Der Auszubildende lernt die drei grundlegenden Sprachen einer Webseite kennen.\\
			\hline
			3. Auflistung der einzelnen Themen. Diese werden zusätzlich auf dem Flipchart übertragen & Der Auszubildende bekommt einen Überblick über die einzelnen Themen die gleich behandelt werden.\\
			\hline
			4. Zur Einordnung der einzelnen Sprachen dient ein Krokodil oder Dino .gif, was die einzelnen Sprachen im wesentlichen in Ihrer Funktionsweise zeigt. & Der Auszubildende ordnet grob die einzelnen Sprachen der Funktionsweise zu.\\
			\hline
			5. Erklärung von Hypertext Markup language & Hier lernt der Auszubildende den Begriff HTML kennen und was HTML ist. \\
			\hline
			6. Die Weiterentwicklung von HTML und wer diese Erfunden hat & Der Auszubildende soll die wichtigen Institutionen kennen und wer HTML erfunden hat.\\
			\hline
			7. Zeigt die erste Webseite in der Welt. Auf Code wurde verzichtet, da dieser selbst Fehler aufweist.  & Der Auszubildende soll die erste Webseite kennen lernen. \\
			\hline
			8. Wie ist HTML aufgebaut. & Der Lernende bekommt einen Einblick in den grundlegenden Aufbau von HTML und wie ein Tag in HTML aufgebaut ist.\\
			\hline
			9.Einblick in Cascading Style Sheets & Der Lehrling lernt kenne was CSS ist, wer es weiterentwickelt und wer es erfunden hat. \\
			\hline
			10. Codebeispiel für CSS mit html & Der Auszubildende lernt den "Einbau" von CSS in HTML kennen. Daneben wird der Effekt sichtbar nachvollziehbar.\\
			\hline
			11. Hier ist ein Codeschnipsel wie nur CSS aufgebaut ist. & Der Lehrling lernt die Syntax kennen.\\
			\hline
			12. Es wird auf JavaScript eingegangen. & Der Auszubildende bekommt einen Einblick auf die Möglichkeiten von JavaScript und dessen Erfinder.\\
			\hline
			13.Hier wird Anhang eines Beispiels JavaScript demonstriert. & Der Lehrling bekommt einen Eindruck davaon, was JavaScript mit HTML anrichten kann.\\
			\hline
			\end{tabular}
	\end{center}
\end{table}
\newpage
\begin{table}[h!]
	\begin{center}
		\caption{Inhaltsvermittlung 2}
		\label{tab:table1}
		\begin{tabular}{|p{8cm}|p{8cm}|}
			\textbf{Präsentationsthemen} & \textbf{Zweck}\\
			\hline
			14. Das Beispiel zuvor wird Anhand diesen Code erklärt. & Der Auszubildende lernt die Einbindung von JavaScript in HTML kennen. Hinzu kommt, wie in JavaScript eine Methode aufgebaut ist.\\
			\hline
			15. Die Validatoren werden gezeigt & Der Auszubildende lernt die Validationsmöglichkeiten kennen und weiß wonach er er suchen muss und das es welche gibt. Dies steigert die Codequalität.\\
			\hline
			16. Als nächstes wird auf den Ausblick eingegangen, was hier nur in kleinem Rahmen dargestellt wird. & Der Auszubildende weiß nun, dass es bei den grundlegenden Sprachen einer Webseite nicht bleibt und kann sich hier neues wissen aneignen. Bestes Beispiel ist diese Präsentation.\\
			\hline
			17. Letzte Folie allgemeine Hinweise und Questions und Answers & ""\\
			\hline
		\end{tabular}
	\end{center}
\end{table}

\begin{itemize}
	\item Kernaussagen zusammengefasst\\
	\subitem Aufbau von HTML
	\subitem Aufbau von CSS
	\subitem Aufbau JavaScript
	\subitem Kombinieren der drei grundlegenden Sprachen einer Webseite
	\subitem Steigerung der Codequalität
	\subitem Weitere Möglichkeiten auf Basis HTML, CSS, JavaScript
\end{itemize}

\section{Schluss}
	\begin{itemize}
		\item Erfolgssicherung\\
		Der Lehrling wird aufgefordert einen Eintrag ins Berichtsheft zu verfassen und die Questions und Answers im Ausbildungsnachweis zu verwenden. Es können möglich auftretende Fragen beantwortet werden.
		\item Bezug zur Praxis
		Dem Auszubildenden wird verdeutlicht, dass es HTML, CSS, JavaScript allgegenwärtig ist. In Präsentationstechniken, Github z.B. Markdown, Beispiele in Codepen.io. Jede Webseite ist damit aufgebaut.
		\item Anlage\\
		Ausschnitt aus dem Ausbildungsrahmenplan
	\end{itemize}
\newpage

%\includepdf[pages={1}]{myfile.pdf}
\newpage

\section{Glossar}
   \begin{table}[h!]
   	\caption{Glossar}
   	\label{tab:table1}
   	\begin{tabular}{|l|p{10cm}|}
   		\textbf{Fremdwort} & \textbf{Erklärung}\\
   		\hline
   		operationalisiert &[1] theoretische Begriffe und Hypothesen im Sinn ihrer empirischen Überprüfbarkeit umformulieren, so dass im Einzelnen überprüfbare Zielvorgaben und Schritte vorliegen\\
   		\hline
   		Hypertext Markup Language & Hypertext-Auszeichnungssprache\\
   		\hline
   		Cascading Style Sheets & für gestufte Gestaltungsbögen\\
   		\hline
   	\end{tabular}
\end{table}

\end{document}

      