\documentclass[a4paper]{exam}
\newcommand{\Antwort}[1]{
	\fillin[#1]\hfill\qrcode{#1}\vspace{2em}
}
\usepackage{qrcode} % keine Umlaute verwende ä, ü, ö, 
\usepackage{listings}

\begin{document}
	\section{HTML}
		\begin{itemize}
			\item Was heißt HTML?
			\Antwort{Hypertext\ Markup\ Language,\ Kaskadierende Stilvorlagen}
			%\fillin[Hypertext Markup Language] \hfill\qrcode{Hypertext Markup %Language}\vspace{2em}
			
			\item Welche Art von Auszeichnungsprache ist es? %\fillin[texbasiert]\hfill\qrcode{textbasiert}\vspace{2em}\\
			\Antwort{textbasiert}
			
			\item Für was wird Sie verwendet?
			%\fillin[zur Strukturierung] \hfill \qrcode{zur Strukturierung}\vspace{2em}
			\Antwort{zur\ Strukturierung}
			
			\item Für welche Dokumente wird HTML verwendet?
			%\fillin[elektronische Dokumente] \hfill \qrcode{elektronische Dokumente}\vspace{2em}
			\Antwort{elektronische\ Dokumente}
			
			\item Was heißt WWW?
			%\fillin[World Wide Web] \hfill \qrcode{World Wide Web}\vspace{2em}
			\Antwort{World\ Wide\ Web}
			
			\item Gehört es zu den Grundlagen des WWW?
				\begin{oneparcheckboxes}
					\newline
					\choice yes
					\choice no
					\hfill\qrcode{yes\ }\vspace{2em}
				\end{oneparcheckboxes}	
		  	
	
			\item Was ist ein Webbrowser?
			%\fillin[ Computerprogramme zur Darstellung von Webseiten] \hfill \qrcode{Computerprogramme zur Darstellung von Webseiten}\vspace{2em}
			\Antwort{Ein\ Computerprogramm\ zur\ Darstellung\ einer\ Webseite}
			
			\item Welche zwei gängige Dokumentenarten lassen sich darstellen?\\
			\fillin[pdf] \fillin[html] \hfill \qrcode{pdf, html}\vspace{2em}
			
			\item Welche Aufgabe hat HTML?
			%\fillin[Text semantisch zu strukturieren] \hfill \qrcode{Text semantisch zu strukturieren}\vspace{2em}
			\Antwort{Text\ semantisch\ zu\ Strukturieren}
			
			\item Welche Matainformationen fallen Ihnen ein?
			%\fillin[Sprache] \fillin[Autor] \hfill \qrcode{Sprache, Autor}
			
			\item Wer entwickelt HTML weiter?
			\fillin[W3C] \fillin[WHATWG] \hfill \qrcode{W3C, WHATWG}
			
			\item Wer hat Sie erfunden?
			%\fillin[Europäischen Organisation für Kernforschung, CERN] \hfill \qrcode{Europaeischen Organisation fuer Kernforschung}
			\Antwort{Europaeischen\ Organisation\ fuer\ Kernforschung, CERN}
			
			\item Wann wurde Sie erfunden?
			\Antwort{1989}
			\item Finde den Fehler!
		\end{itemize}
		\begin{lstlisting}[language=html]
<html>
    <head>
        <title>Page Title</title>
    </head>
    <body>
    
        <h1>This is a Heading</h1>
        <p>This is a paragraph.</p>
        
    </body>
</html> 
		\end{lstlisting}
		\vfill
		\qrcode{https://www.w3schools.com/html/tryit.asp?filename=tryhtml_intro}
		\newpage
		
		\section{CSS}
			\begin{itemize}
				\item Was heißt CSS?
				\Antwort{Cascading Style Sheets, Kaskadierende Stilvorlagen} 
				
				\item Für Welche Art von Dokumenten ist die Sprache gedacht?
				\Antwort{Fuer\ elektronische\ Dokumente}
				
				\item Für Welche Erstellung ist Sie gedacht?
				\Antwort{Fuer\ Gestaltungsanweisungen}
				
				\item Wer hat Sie erfunden?
				\Antwort{Hakon\ Wium\ Lie, CERN}
				
				\item Wann wurde Sie erfunden?
				\Antwort{1996}
				
				\item In welchem Tag befindet Sie sich?
				\Antwort{<style>}
				\newpage
				\item Finde den Fehler!
			\end{itemize}

				\begin{lstlisting}
<!DOCTYPE html>
	<html>
		<head>
			<style>
				body {
					background-color: lightblue;
				}

				h1 {
					color: white;
					text-align: center;
				}

				p {
					font-family: verdana;
					font-size: 20px;
				}
		</head>
		<body>

			<h1>My First CSS Example</h1>
			<p>This is a paragraph.</p>

		</body>
	</html>
				\end{lstlisting}
				\vfill
\qrcode{https://www.w3schools.com/css/tryit.asp?filename=trycss_default}
				\newpage
				\section{JavaSript}
			\begin{itemize}
				\item Was für eine Sprache ist JavaScript?
				\Antwort{Eine\ Srcriptsprache}
				
				\item Für was wurde Sie ursprünglich entwickelt
				\Antwort{Um Benutzerinteraktionan\ auszuwerten}
				
				\item Was macht JavaScript mit Inhalten?
				\Antwort{Inhalte Veraendern,\ nachladen,\ generieren}
				
				\item Wieso nicht LiveScript?
				\Antwort{Um die Popularitaet\ von\ Java\ zu\ nutzen}
				
				\item welche Besonderheiten weißt Javascript in der Sprach noch auf?\\
				\Antwort{Sie\ ist\ objektorientiert,\ prozedural,\ funktional}
				
				
				
			\end{itemize}
		\begin{checkboxes}
			\choice test
		\end{checkboxes}
		
		 
		%\(12345 + 67890 = \) \fillin[80235] \hfill\qrcode{80235}\vspace{2em} \\
		123456+789 =\fillin[123456] \hfill \qrcode{test}
\end{document}
